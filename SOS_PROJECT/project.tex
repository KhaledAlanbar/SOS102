\documentclass{article}
\usepackage[utf8]{inputenc}
\usepackage{times}
\usepackage{setspace}
\usepackage[margin=1in]{geometry}
\usepackage{natbib}
\usepackage{hyperref}
\usepackage{array}
\usepackage{graphicx} 
\usepackage{titling} 
\usepackage{indentfirst} 
\usepackage{setspace}
\doublespacing

\pretitle{
  \begin{center}
  \includegraphics[width=0.4\textwidth]{YULogo.png}\\[\bigskipamount]
  \Large\bfseries
}
\title{SOS 102: Semester Assignment\\
\large Can people live without social media these days?\\
\vspace{0.5cm}
\normalsize Submitted to: Dr. Rihab Alsaeed}
\posttitle{\end{center}}
\date{\today}

\begin{document}

\maketitle

\vspace{1cm}
\begin{center}
\begin{tabular}{|p{5cm}|p{5cm}|}
\hline
\textbf{Student Name} & \textbf{Student ID} \\
\hline
Khaled Alanbar & 202211365 \\
\hline
Rian Bawazeer & 202111145 \\
\hline
Kenan Char  & 202211085 \\
\hline
\end{tabular}
\end{center}
\newpage

\begin{abstract}
\noindent
In today's interconnected world, social media has become deeply integrated into our daily lives, raising questions about whether modern society can function without it. This research paper examines the possibility of living without social media through three sociological perspectives: functionalist, conflict, and interactionist approaches. By analyzing current research, social patterns, and real-world examples, we explore the dependencies, challenges, and potential alternatives to social media in contemporary society.
\end{abstract}
\newpage

\section{Introduction}
\indent Social media has revolutionized how we communicate, work, and maintain relationships in the 21st century. With billions of active users across various platforms, it has become an integral part of modern life, influencing everything from personal connections to business operations. This paper investigates the critical question: "Can people live without social media these days?"

Our analysis employs three major sociological perspectives to examine this question comprehensively. We will explore the structural functions of social media in society, analyze power dynamics and inequalities in digital spaces, and examine how individuals interact and create meaning through social media platforms.

\section{Functionalist Perspective (Kenan)}
The functionalist perspective views social media as essential infrastructure that maintains societal stability by facilitating connections, communication, and shared norms in modern life.

\subsection{Social Integration and Communication}
Social media has become so deeply embedded in our daily lives that it's difficult to imagine functioning without it. From a sociological perspective, particularly through the functionalist lens, these platforms serve crucial roles in maintaining social order and cohesion \citep{reitz2012}. They've evolved beyond simple communication tools into complex social institutions that facilitate connection, information sharing, economic activity, and cultural exchange on an unprecedented scale. The connective power of social media represents one of its most significant societal functions. Platforms like Facebook, WhatsApp, and Instagram have transformed how we maintain relationships across distances. Where international communication was once limited and expensive, these tools now enable instant, free interaction regardless of location \citep{reitz2012}. This became particularly vital during the COVID-19 pandemic when physical distancing measures made digital connections essential for maintaining social bonds. Beyond personal relationships, professional networks like LinkedIn demonstrate how these platforms create economic opportunities by connecting job seekers with employers and facilitating career growth through digital networking.

\subsection{Education and Knowledge Dissemination}
When examining information dissemination, social media's role in education and knowledge-sharing stands out as particularly transformative. The pandemic highlighted this function when schools worldwide rapidly adopted platforms like Zoom and Microsoft Teams to continue instruction \citep{onyema2020}. Beyond formal education, these platforms have revolutionized how we access and share information. News now breaks first on Twitter, scientific discussions happen in real-time through academic Facebook groups, and YouTube serves as an endless repository of tutorials and educational content \citep{hilton2016}. This democratization of knowledge represents a fundamental shift in how information flows through society, making expertise more accessible than ever before.

\subsection{Economic and Cultural Impact}
The economic impact of social media cannot be overstated, particularly for small businesses and entrepreneurs. These platforms have leveled the playing field, allowing local businesses to reach thousands of potential customers through Instagram posts or TikTok videos \citep{reitz2012}. Customer service has transformed from frustrating phone trees to real-time Twitter interactions, while crowdfunding platforms enable innovative ideas to find financial support directly from interested communities. This economic function has created new opportunities and changed traditional business models in ways that benefit both consumers and producers. Culturally, social media serves as a global public square where ideas, trends, and movements spread with remarkable speed. The Black Lives Matter movement for example, gained international traction through these platforms, demonstrating their power to drive social change. Memes and viral challenges create shared cultural experiences that transcend geographic boundaries, while exposure to diverse perspectives fosters greater understanding between different groups. For younger generations especially, these platforms play a significant role in socialization, shaping how they learn behavioral norms and engage with the world around them. While social media presents well-documented challenges including misinformation, addictive design, and privacy concerns, these issues don't negate its fundamental value to modern society. Instead, they highlight the need for thoughtful regulation and digital literacy education to maximize benefits while minimizing harms. The reality remains that billions of people worldwide now rely on these platforms for work, education, social connection, and entertainment. Traditional alternatives simply can't match the speed, reach, and accessibility that digital platforms provide.

From this perspective, social media has become as fundamental to modern life as electricity or running water - we don't just use it, we depend on it to function as a society. As these platforms keep changing and growing, recognizing how they actually help us connect and share information makes it easier to value their good sides while fixing their problems. The real trick now is keeping all the amazing benefits we've come to rely on, while finding smart ways to handle the downsides, so these tools keep working for all of us in the long run.

\newpage
\section{Conflict Perspective (Khaled)}
The conflict perspective helps us understand how power and inequality work in society. When we look at social media this way, we can see how it creates differences between people and how big companies control our online lives. This view shows us who has power, who doesn't, and how social media makes these differences bigger.

\subsection{The Digital Gap}
Not everyone has the same access to social media. Think about your grandparents who might struggle with technology, or people living in poor areas who can't afford smartphones. According to \citet{ye2023}, this creates a big divide in society between those who can use social media and those who can't. Some people are left out because they can't afford smartphones, computers, or internet service. Others live in areas with poor or no internet connection. Many don't have the skills or education to use these tools, while some might be too old or uncomfortable with new technology. Additionally, some people speak languages that aren't well-supported on social media.

This gap means some people get left behind in today's digital world. For example, during the COVID-19 pandemic, people without social media missed important updates and felt more isolated than others.

\subsection{Big Companies and Control}
Large companies like Meta (Facebook), Twitter (X now), and Instagram have enormous power over our daily lives. \citet{azizi2019} explains that these companies are like digital landlords, they own the spaces where we connect with friends, share our thoughts, and get our news. These companies exercise their control in several ways: they decide what posts you see in your feed, track everything you do online, and sell your personal information to advertisers. They can change their rules whenever they want and can delete your account or posts without much explanation.

Think about it like this: imagine if one company owned all the parks, cafes, and meeting places in your city. That's similar to how these social media companies control our online spaces.

\subsection{Pressure to Use Social Media}
Many people feel forced to use social media even if they don't want to. It's like having a phone number these days you might not love having one, but it's hard to live without it. People feel pressured because many jobs opportunities are open their only on LinkedIn. Invitations (Family invitation) and event planning often happen through Whatsapp or other apps. Family members share important updates on WhatsApp. Schools and teachers communicate through some apps, for example: telegram, emails, or whatsapp groups. Local businesses share deals and updates on social platforms, and community groups organize activities through social media.

\subsection{Great Things About Not Using Social Media}
Living without social media isn't all bad. There are several benefits to consider: you get more privacy in your daily life and spend less time wasted scrolling through feeds (Like me when I start scrolling through TikTok all day). People often experience better mental health with no more comparing yourself to others, and develop more meaningful face to face friendships. You gain freedom from constant notifications and distractions (When Studying), avoid pop up or dedicated ads following you around, and have more time for hobbies and physical activities.

\subsection{Difficult Things About Not Using Social Media}
However, staying off social media comes with real challenges. It becomes harder to find out about job opportunities, like those on linkdein as mentioned in pressure to use social media. People end up missing family photos and updates, in trips or special moments (e.g. Birthdays). There's difficulty staying up-to-date with local news, and missing discounts and special offers. It becomes harder to research businesses or services (e.g. Tawakkalna,Sehhaty,Noon,Amazon), and many end up losing touch or contact with old friends.

\newpage
\section{Interactionist Perspective (Rian)}
From an interactional perspective, technology, especially in the context of social media, is deeply embedded in everyday human interactions today. Furthermore, social media now facilitates the sharing of a wide variety of symbols that people understand, including images, language, or specific icons. When people turn away from technology and social media, they feel isolated because they have lost one of the places where they share the symbols and everyday interactions that technology has facilitated.

\subsection{Open-Source Science and Social Media in Education}
Social media contributes to the development of education, especially in the areas of open-source science. Much science is now available online to everyone, and all groups can easily access it because it is cheap or free due to its open-source nature \citep{siddiqi2024}. From the author's perspective, open-source science on social media greatly facilitates rapid learning, as scientists can pick up where others left off without having to start from scratch \citep{hilton2016}. Most programmers in the fields of technology science publish their work for others to benefit from \citep{hilton2016}. Publishing medical hypotheses and research also helps people in the same field communicate by writing a comment or share a posts and stay on the same scientific level. Despite the many benefits, there are some risks of leaking some data in the social media, For example, sciences that have not been fully tested may spread and be circulated by people in the real world and they begin to use them. Ultimately, social media has helped advance education because open source science sharing on social media has helped accelerate learning.

\subsection{Social Media's Role in Education During the COVID-19 Pandemic}
When the Corona pandemic struck, the importance of social media in people's lives became clear, especially in the context of ensuring that education does not stop during the pandemic. After the emergence of cases of coronavirus and the significant increase in numbers, schools were closed in more than 100 countries \citep[cited in][]{onyema2020}. This closure affected education for students, especially young students who were accustomed to traditional education. As \citet[p.1]{onyema2020} highlight, ``COVID-19 school closures left over one billion learners out of school.'' Many educational institutions adopted e-learning in schools to continue their studies, using virtual education platforms which is considered a social media such as Teams \citep{onyema2020}. In the author's opinion, e-learning using social media platform was the best option for completing education and minimizing the risks as much as possible. There are many advantages to the introduction of social media into the world of education, especially during the COVID-19 pandemic. The first advantage was flexibility in education. Technology made it easier for students to attend classes from anywhere, helping them manage their time and balance their personal and professional lives \citep{moise2021}. Secondly, Microsoft services provided easy-to-use, integrated learning organizations that made it easier for students and faculty members to communicate with each other via calls or messages. Thus, social media technology enhanced communication during education, even in the most difficult circumstances \citep{moise2021}. Thirdly, the future of education is blended learning, which combines traditional learning with learning using social media to communicate remotely even when we have force majeure \citep{moise2021}. In the end, people will not be able to live without social media, because education will stop in the event of any emergency.
\newpage
\section{Conclusion}
After looking at social media from different angles, we can see that living without it today is possible but very difficult. Like electricity or phones, social media has become a basic part of how our society works. From the functionalist view, we saw how it helps people stay connected, learn, and do business. It's especially important during times like the COVID-19 pandemic, where it kept education going and people connected. The conflict view showed us some problems, like how not everyone has equal access to social media, and how big companies control these platforms. But it also reminded us that taking breaks from social media can be good for our mental health and real-life relationships. Looking at how people interact, especially in education, we saw that social media isn't just a tool anymore, it's changed how we talk to each other, learn, and share information. The real challenge isn't about whether we can live without social media, but about finding better ways to use it while dealing with its problems.

\newpage
\bibliographystyle{apalike}
\bibliography{references}

\end{document}
